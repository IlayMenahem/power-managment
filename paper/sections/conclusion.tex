\section{Conclusion}
\label{sec:conclusion}

The paper proposed a new \emph{End-to-End Learning and Repair} (E2ELR)
architecture for training optimization proxies for economic dispatch
problems. E2ELR combines deep learning with closed-form, differential
repair layers, thereby integrating prediction and feasibility
restoration in an end-to-end fashion. The E2ELR architecture can be
trained with self-supervised learning, removing the need for
labeled data and the solving of numerous optimization problems
offline. The paper conducted extensive numerical experiments on the
ecocomic dispatch of large-scale, industry-size power grids with tens
of thousands of buses.  It also presented the first study that
considers reserve requirements in the context of optimization proxies,
reducing the gap between academic and industry formulations.
The results demonstrate that the combination of E2ELR and
self-supervised learning achieves state-of-the-art performance, with
optimality gaps that outperform other baselines by at least an order
of magnitude.  Future research will investigate security-constrained
economic dispatch (SCED) formulations, and the extension of repair
layers to \revision{}{thermal constraints, multi-period settings and the nonlinear, non-convex AC-OPF.}
