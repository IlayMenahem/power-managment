\section{Numerical Results}
\label{sec:results}

\subsection{Optimality Gaps}
\label{sec:results:gaps}

Table \ref{tab:res:opt_gap} reports, for the ED and ED-R problems, the
mean optimality gap of each ML model, under the SL and SSL learning
modes. Bold entries denote the best-performing method. Recall that
LOOP is not included in ED-R experiments. {\em E2ELR systematically
  outperforms all other baselines across all settings.}  This stems
from two reasons. First, DNN, \revision{}{DeepOPF} and DC3 exhibit violations of the power
balance constraint \eqref{eq:DCOPF:power_balance}, which yields high
penalties and therefore large optimality gaps.  Statistics on power
balance violations for DNN, \revision{}{DeepOPF} and DC3 are reported in Table
\ref{tab:res:violation}. Second, LOOP's poor performance, despite
not violating any hard constraint, is because the non-convex gauge mapping
used inside the model has an adverse impact on training.
Indeed, after a few epochs of training, LOOP gets stuck
in a local optimum.

    \begin{table}[!t]
        \centering
        \caption{Mean optimality Gap (\%) on test set}
        \label{tab:res:opt_gap}
        \resizebox{\columnwidth}{!}{
        \begin{tabular}{llrrrrrrrrr}
\toprule
&
    & \multicolumn{5}{c}{ED}
    & \multicolumn{4}{c}{ED-R}\\
\cmidrule(lr){3-7}
\cmidrule(lr){8-11}
Loss & System         
        &      DNN &    E2ELR &      \revision{}{DeepOPF} & DC3 &     LOOP
        &      DNN &    E2ELR &      \revision{}{DeepOPF} & DC3 \\
\midrule
SL   & \ieeeSmall     
        &    69.55 &     \textbf{1.42} &           \revision{}{2.81} &     3.03 &    38.93
        &    75.06 &     \textbf{1.52} &           \revision{}{2.80} &     2.94\\
     & \pegaseSmall   
        &    48.77 &     \textbf{0.74} &           \revision{}{7.78} &     2.80 &    32.53
        &    47.84 &     \textbf{0.74} &           \revision{}{7.50} &     2.97\\
     & \rteLarge           
        &    55.13 &     \textbf{1.35} &          \revision{}{28.23} &     3.68 &    50.21
        &    70.57 &     \textbf{1.82} &          \revision{}{46.66} &     3.49\\
     & \pegaseLarge        
        &    76.06 &     \textbf{0.38} &          \revision{}{33.20} &     1.25 &    33.78
        &    81.19 &     \textbf{0.38} &          \revision{}{30.84} &     1.29\\
     & \pegaseXLarge       
        &    71.14 &     \textbf{0.29} &          \revision{}{64.93} &     1.79 &    52.94
        &    76.32 &     \textbf{0.28} &          \revision{}{69.23} &     1.81\\
     & \gocXXL             
        &   194.13 &     \textbf{0.46} &          \revision{}{55.91} &     2.75 &    36.49
        &   136.25 &     \textbf{0.45} &          \revision{}{41.34} &     2.35\\
\midrule
SSL  & \ieeeSmall          
        &    35.66 &     \textbf{0.74} &          \revision{}{ 2.23} &     2.51 &    37.78
        &    45.56 &     \textbf{0.78} &          \revision{}{ 2.82} &     2.80\\
     & \pegaseSmall        
        &    62.07 &     \textbf{0.63} &          \revision{}{10.83} &     2.57 &    32.20
        &    64.69 &     \textbf{0.68} &          \revision{}{ 9.83} &     2.61\\
     & \rteLarge           
        &    40.73 &     \textbf{1.30} &          \revision{}{42.28} &     2.82 &    50.20
        &    55.16 &     \textbf{1.68} &          \revision{}{48.57} &     3.04\\
     & \pegaseLarge        
        &    43.68 &     \textbf{0.32} &          \revision{}{34.33} &     0.82 &    33.76
        &    44.74 &     \textbf{0.29} &          \revision{}{42.06} &     0.93\\
     & \pegaseXLarge       
        &    57.58 &     \textbf{0.21} &          \revision{}{60.12} &     0.84 &    52.93
        &    61.28 &     \textbf{0.19} &          \revision{}{65.38} &     0.91\\
     & \gocXXL             
        &   108.91 &     \textbf{0.39} &          \revision{}{ 8.39} &     0.72 &    36.73
        &    93.91 &     \textbf{0.33} &          \revision{}{ 9.47} &     0.71\\     
\bottomrule
\end{tabular}

        }\\
    \end{table}

        \begin{table}[!t]
        \centering
        \caption{Power balance constraint violation statistics}
        \label{tab:res:violation}
        \resizebox{\columnwidth}{!}{%
            % constraint violation (abs)
\begin{tabular}{llrrrrrrrrrrrr}
\toprule
&
    & \multicolumn{6}{c}{ED}
    & \multicolumn{6}{c}{ED-R}\\
\cmidrule(lr){3-8}
\cmidrule(lr){9-14}
     &               
     & \multicolumn{2}{c}{DNN}
     & \multicolumn{2}{c}{\revision{}{DeepOPF}}
     & \multicolumn{2}{c}{DC3}
     & \multicolumn{2}{c}{DNN}
     & \multicolumn{2}{c}{\revision{}{DeepOPF}}
     & \multicolumn{2}{c}{DC3}\\
\cmidrule(lr){3-4}
\cmidrule(lr){5-6}
\cmidrule(lr){7-8}
\cmidrule(lr){9-10}
\cmidrule(lr){11-12}
\cmidrule(lr){13-14}
Loss & System        
    & \multicolumn{1}{c}{\%feas} & \multicolumn{1}{c}{viol}
    & \multicolumn{1}{c}{\%feas} & \multicolumn{1}{c}{viol}
    & \multicolumn{1}{c}{\%feas} & \multicolumn{1}{c}{viol}
    & \multicolumn{1}{c}{\%feas} & \multicolumn{1}{c}{viol}
    & \multicolumn{1}{c}{\%feas} & \multicolumn{1}{c}{viol}
    & \multicolumn{1}{c}{\%feas} & \multicolumn{1}{c}{viol}\\
\midrule
SL   & \ieeeSmall    
         &      0\% &   0.50  &    \revision{}{100\%} &   \revision{}{0.00}  &    100\% &   0.00 
         &      0\% &   0.70  &    \revision{}{100\%} &   \revision{}{0.06}  &    100\% &   0.25 \\
     & \pegaseSmall  
         &      0\% &   1.82  &     \revision{}{71\%} &   \revision{}{1.35}  &     65\% &   0.04 
         &      0\% &   1.90  &     \revision{}{70\%} &   \revision{}{1.19}  &     65\% &   0.08 \\
     & \rteLarge     
         &      0\% &   2.08  &      \revision{}{1\%} &   \revision{}{1.94}  &      9\% &   0.03 
         &      0\% &   2.26  &      \revision{}{1\%} &   \revision{}{2.81}  &      7\% &   0.01 \\
     & \pegaseLarge  
         &      0\% &   6.25  &      \revision{}{3\%} &   \revision{}{6.38}  &     29\% &   0.00 
         &      0\% &   5.91  &      \revision{}{3\%} &   \revision{}{5.99}  &     29\% &   0.00 \\
     & \pegaseXLarge 
         &      0\% &   6.63  &      \revision{}{1\%} &   \revision{}{7.08}  &     23\% &   0.00 
         &      0\% &   7.79  &      \revision{}{1\%} &   \revision{}{7.46}  &     22\% &   0.00 \\
     & \gocXXL       
         &      0\% &   3.81  &     \revision{}{45\%} &   \revision{}{1.70}  &     57\% &   0.03 
         &      0\% &   2.31  &     \revision{}{54\%} &   \revision{}{1.54}  &     69\% &   0.00 \\
\midrule
SSL  & \ieeeSmall    
         &      0\% &   0.58  &    \revision{}{100\%} &   \revision{}{0.00}  &    100\% &   0.16 
         &      0\% &   0.73  &    \revision{}{100\%} &   \revision{}{0.56}  &    100\% &   0.29 \\
     & \pegaseSmall  
         &      0\% &   2.42  &     \revision{}{63\%} &   \revision{}{2.56}  &     63\% &   0.03 
         &      0\% &   2.30  &     \revision{}{63\%} &   \revision{}{1.81}  &     39\% &   0.05 \\
     & \rteLarge     
         &      0\% &   1.90  &      \revision{}{1\%} &   \revision{}{2.64}  &      6\% &   0.11 
         &      0\% &   2.72  &      \revision{}{1\%} &   \revision{}{2.51}  &      5\% &   0.01 \\
     & \pegaseLarge  
         &      0\% &   7.22  &      \revision{}{3\%} &   \revision{}{5.46}  &     27\% &   0.05 
         &      0\% &   7.00  &      \revision{}{3\%} &   \revision{}{7.87}  &     25\% &   0.08 \\
     & \pegaseXLarge 
         &      0\% &   6.41  &      \revision{}{1\%} &   \revision{}{7.19}  &     19\% &   0.01 
         &      0\% &   6.82  &      \revision{}{1\%} &   \revision{}{7.83}  &     20\% &   0.01 \\
     & \gocXXL       
         &      0\% &   3.18  &     \revision{}{55\%} &   \revision{}{1.24}  &     52\% &   0.01 
         &      0\% &   2.61  &     \revision{}{49\%} &   \revision{}{1.20}  &     62\% &   0.02 \\
\bottomrule
\end{tabular}

        }
        \footnotesize{$^{\dagger}$with 200 gradient steps. $^{*}$geometric mean of non-zero violations, in p.u.}
        \end{table}
        

{\em E2ELR, when trained in a self-supervised mode, achieves the best
  performance.}  This is because SSL directly minimizes the true
objective function, rather than the surrogate supervised loss.  With
the exception of \rteLarge, the performance of E2ELR improves as the
size of the system increases, with the lowest optimality gaps achieved
on \pegaseXLarge, which has the most generators.  Note that \rteLarge
is a real system from the French transmission grid: it is more
congested than other test cases, and therefore harder to learn.


\subsection{Computing Times}
\label{sec:results:times}

Tables \ref{tab:exp:ED:train_time} and \ref{tab:exp:ED-R:train_time}
report the sampling and training times for ED and ED-R, respectively.
Each table reports the total time for data-generation, which
corresponds to the total solving time of Gurobi on a single thread.
There is no labeling time for self-supervised models.  While training
times for SL and SSL are comparable, for a given architecture, the
latter does not incur any labeling time.  The training time of DC3 is
significantly higher than other baselines because of its unrolled
gradient steps.  These results demonstrate that ML models can be
trained efficiently on large-scale systems.  Indeed, {\em the
  self-supervised E2ELR needs less than an hour of total computing
  time to achieve optimality gaps under $0.5\%$ for systems with
  thousands of buses.}

    \begin{table}[!t]
        \centering
        \caption{Sampling and training time comparison (ED)}
        \label{tab:exp:ED:train_time}
        \resizebox{\columnwidth}{!}{
        \begin{tabular}{llrrrrrrrrrrr}
            \toprule
            Loss & System & \multicolumn{1}{c}{Sample} & \multicolumn{1}{c}{DNN} & \multicolumn{1}{c}{E2ELR} & \multicolumn{1}{c}{\revision{}{DeepOPF}} &\multicolumn{1}{c}{DC3} & \multicolumn{1}{c}{LOOP}  \\
            \midrule
            SL
            & \ieeeSmall       & \qty[mode=text]{ 0.2}{\hour} & \qty[mode=text]{ 7}{\minute}  & \qty[mode=text]{37}{\minute} & \revision{}{\qty[mode=text]{31}{\minute}} & \qty[mode=text]{121}{\minute} & \qty[mode=text]{ 33}{\minute} \\\
            & \pegaseSmall     & \qty[mode=text]{ 0.7}{\hour} & \qty[mode=text]{ 8}{\minute}  & \qty[mode=text]{14}{\minute} & \revision{}{\qty[mode=text]{ 6}{\minute}} & \qty[mode=text]{ 41}{\minute} & \qty[mode=text]{ 19}{\minute} \\\
            & \rteLarge        & \qty[mode=text]{ 5.1}{\hour} & \qty[mode=text]{11}{\minute}  & \qty[mode=text]{30}{\minute} & \revision{}{\qty[mode=text]{13}{\minute}} & \qty[mode=text]{ 73}{\minute} & \qty[mode=text]{ 18}{\minute}\\
            & \pegaseLarge     & \qty[mode=text]{12.7}{\hour} & \qty[mode=text]{15}{\minute}  & \qty[mode=text]{24}{\minute} & \revision{}{\qty[mode=text]{22}{\minute}} & \qty[mode=text]{123}{\minute} & \qty[mode=text]{ 25}{\minute}\\
            & \pegaseXLarge    & \qty[mode=text]{20.6}{\hour} & \qty[mode=text]{14}{\minute}  & \qty[mode=text]{19}{\minute} & \revision{}{\qty[mode=text]{14}{\minute}} & \qty[mode=text]{126}{\minute} & \qty[mode=text]{ 19}{\minute}\\
            & \gocXXL          & \qty[mode=text]{63.4}{\hour} & \qty[mode=text]{25}{\minute}  & \qty[mode=text]{20}{\minute} & \revision{}{\qty[mode=text]{41}{\minute}} & \qty[mode=text]{108}{\minute} & \qty[mode=text]{127}{\minute}\\
            \midrule
            SSL
            & \ieeeSmall       & -- &  \qty[mode=text]{15}{\minute} & \qty[mode=text]{27}{\minute} & \revision{}{\qty[mode=text]{38}{\minute}} & \qty[mode=text]{102}{\minute} & \qty[mode=text]{27}{\minute} \\
            & \pegaseSmall     & -- &  \qty[mode=text]{ 8}{\minute} & \qty[mode=text]{15}{\minute} & \revision{}{\qty[mode=text]{11}{\minute}} & \qty[mode=text]{ 46}{\minute} & \qty[mode=text]{14}{\minute} \\
            & \rteLarge        & -- &  \qty[mode=text]{ 9}{\minute} & \qty[mode=text]{17}{\minute} & \revision{}{\qty[mode=text]{10}{\minute}} & \qty[mode=text]{ 42}{\minute} & \qty[mode=text]{15}{\minute} \\
            & \pegaseLarge     & -- &  \qty[mode=text]{18}{\minute} & \qty[mode=text]{20}{\minute} & \revision{}{\qty[mode=text]{17}{\minute}} & \qty[mode=text]{100}{\minute} & \qty[mode=text]{29}{\minute} \\
            & \pegaseXLarge    & -- &  \qty[mode=text]{17}{\minute} & \qty[mode=text]{18}{\minute} & \revision{}{\qty[mode=text]{14}{\minute}} & \qty[mode=text]{125}{\minute} & \qty[mode=text]{15}{\minute} \\
            & \gocXXL          & -- &  \qty[mode=text]{38}{\minute} & \qty[mode=text]{45}{\minute} & \revision{}{\qty[mode=text]{51}{\minute}} & \qty[mode=text]{105}{\minute} & \qty[mode=text]{60}{\minute} \\
            \bottomrule
        \end{tabular}
        }
        \footnotesize{Sampling (training) times are for 1 CPU (1 GPU). Excludes hypertuning.}
    \end{table}

    \begin{table}[!t]
        \centering
        \caption{Sampling and training time comparison (ED-R)}
        \label{tab:exp:ED-R:train_time}
        \resizebox{\columnwidth}{!}{
        \begin{tabular}{llrrrrr}
            \toprule
            Loss & System & \multicolumn{1}{c}{Sample} & \multicolumn{1}{c}{DNN} & \multicolumn{1}{c}{E2ELR} & \multicolumn{1}{c}{\revision{}{DeepOPF}} & \multicolumn{1}{c}{DC3} \\
            \midrule
            SL
            & \ieeeSmall       & \qty[mode=text]{ 0.2}{\hour} & \qty[mode=text]{12}{\minute}  & \qty[mode=text]{43}{\minute} & \revision{}{\qty[mode=text]{43}{\minute}} & \qty[mode=text]{115}{\minute} \\
            & \pegaseSmall     & \qty[mode=text]{ 0.8}{\hour} & \qty[mode=text]{14}{\minute}  & \qty[mode=text]{19}{\minute} & \revision{}{\qty[mode=text]{19}{\minute}} & \qty[mode=text]{53 }{\minute} \\
            & \rteLarge        & \qty[mode=text]{ 4.6}{\hour} & \qty[mode=text]{14}{\minute}  & \qty[mode=text]{19}{\minute} & \revision{}{\qty[mode=text]{19}{\minute}} & \qty[mode=text]{71 }{\minute} \\
            & \pegaseLarge     & \qty[mode=text]{14.0}{\hour} & \qty[mode=text]{15}{\minute}  & \qty[mode=text]{22}{\minute} & \revision{}{\qty[mode=text]{22}{\minute}} & \qty[mode=text]{123}{\minute} \\
            & \pegaseXLarge    & \qty[mode=text]{22.7}{\hour} & \qty[mode=text]{16}{\minute}  & \qty[mode=text]{27}{\minute} & \revision{}{\qty[mode=text]{27}{\minute}} & \qty[mode=text]{126}{\minute} \\
            & \gocXXL          & \qty[mode=text]{65.9}{\hour} & \qty[mode=text]{32}{\minute}  & \qty[mode=text]{39}{\minute} & \revision{}{\qty[mode=text]{38}{\minute}} & \qty[mode=text]{129}{\minute} \\
            \midrule
            SSL
            & \ieeeSmall       & -- & \qty[mode=text]{21}{\minute}  & \qty[mode=text]{37}{\minute} & \revision{}{\qty[mode=text]{37}{\minute}} & \qty[mode=text]{131}{\minute} \\
            & \pegaseSmall     & -- & \qty[mode=text]{ 6}{\minute}  & \qty[mode=text]{19}{\minute} & \revision{}{\qty[mode=text]{19}{\minute}} & \qty[mode=text]{ 67}{\minute} \\
            & \rteLarge        & -- & \qty[mode=text]{12}{\minute}  & \qty[mode=text]{21}{\minute} & \revision{}{\qty[mode=text]{21}{\minute}} & \qty[mode=text]{ 71}{\minute} \\
            & \pegaseLarge     & -- & \qty[mode=text]{20}{\minute}  & \qty[mode=text]{24}{\minute} & \revision{}{\qty[mode=text]{24}{\minute}} & \qty[mode=text]{123}{\minute} \\
            & \pegaseXLarge    & -- & \qty[mode=text]{13}{\minute}  & \qty[mode=text]{22}{\minute} & \revision{}{\qty[mode=text]{22}{\minute}} & \qty[mode=text]{125}{\minute} \\
            & \gocXXL          & -- & \qty[mode=text]{52}{\minute}  & \qty[mode=text]{53}{\minute} & \revision{}{\qty[mode=text]{53}{\minute}} & \qty[mode=text]{128}{\minute} \\
            \bottomrule
        \end{tabular}
        }\\
        \footnotesize{Sampling (training) times are for 1 CPU (1 GPU). Excludes hypertuning.}
    \end{table}

Tables \ref{tab:exp:ED:inference_time} and
\ref{tab:exp:ED-R:inference_time} report, for ED and ED-R,
respectively, the average solving time using Gurobi (GRB) and average
inference times of ML methods.  Recall that the Gurobi's solving times
are for a single instance solved on a single CPU core, whereas the ML
inference times are reported for a batch of 256 instances on a GPU.
Also note that the number of gradient steps used by DC3 to recover
feasibility is set to 200 for inference (compared to 50 for training).

On systems with more than 6,000 buses, DC3 is typically 10--30 times
slower than other baselines, again due to its unrolled gradient steps.
\revision{blue}{In contrast, DNN, DeepOPF, E2ELR, and LOOP all require in the
order of 5--10 milliseconds to evaluate a batch of 256 instances.}  For
the largest systems, this represents about 25,000 instances per second,
on a single GPU.  Solving the same volume of instances with Gurobi would require more than
a day on a single CPU.
Getting this time down to the order of seconds, thereby matching the speed of ML proxies, would require thousands of
CPUs, which comes at high financial and environmental costs.

    \begin{table}[!t]
        \centering
        \caption{Solving and inference time comparison (ED)}
        \label{tab:exp:ED:inference_time}
        \resizebox{\columnwidth}{!}{
            \begin{tabular}{llrrrrrrr}
\toprule
Loss & System         &      DNN &    E2ELR &   \revision{}{DeepOPF} &  DC3$^{\dagger}$ &   LOOP & GRB$^{*}$\\
\midrule
SL 
    & \ieeeSmall
        & \qty[mode=text]{3.4}{\ms}
        & \qty[mode=text]{4.5}{\ms}
        & \revision{}{\qty[mode=text]{4.8}{\ms}}
        & \qty[mode=text]{15.4}{\ms}
        & \qty[mode=text]{5.3}{\ms}
        & \qty[mode=text]{12.1}{\ms} \\
    & \pegaseSmall 
        & \qty[mode=text]{4.1}{\ms}
        & \qty[mode=text]{5.3}{\ms} 
        & \revision{}{\qty[mode=text]{4.3}{\ms}}
        & \qty[mode=text]{18.3}{\ms} 
        & \qty[mode=text]{5.9}{\ms} 
        & \qty[mode=text]{51.5}{\ms} \\
    & \rteLarge 
        & \qty[mode=text]{5.1}{\ms} 
        & \qty[mode=text]{6.6}{\ms} 
        & \revision{}{\qty[mode=text]{5.4}{\ms}}
        & \qty[mode=text]{35.3}{\ms} 
        & \qty[mode=text]{7.1}{\ms} 
        & \qty[mode=text]{364.4}{\ms} \\
    & \pegaseLarge 
        & \qty[mode=text]{6.0}{\ms} 
        & \qty[mode=text]{7.3}{\ms} 
        & \revision{}{\qty[mode=text]{6.2}{\ms}}
        & \qty[mode=text]{91.5}{\ms} 
        & \qty[mode=text]{8.2}{\ms} 
        & \qty[mode=text]{913.5}{\ms} \\
    & \pegaseXLarge 
        & \qty[mode=text]{7.3}{\ms} 
        & \qty[mode=text]{8.3}{\ms} 
        & \revision{}{\qty[mode=text]{8.7}{\ms}}
        & \qty[mode=text]{523.6}{\ms} 
        & \qty[mode=text]{13.9}{\ms} 
        & \qty[mode=text]{1481.3}{\ms} \\
    & \gocXXL 
        & \qty[mode=text]{9.5}{\ms} 
        & \qty[mode=text]{10.0}{\ms} 
        & \revision{}{\qty[mode=text]{9.3}{\ms}}
        & \qty[mode=text]{443.0}{\ms} 
        & \qty[mode=text]{14.4}{\ms} 
        & \qty[mode=text]{4566.9}{\ms} \\
\midrule
SSL 
    & \ieeeSmall 
        & \qty[mode=text]{3.4}{\ms} 
        & \qty[mode=text]{6.0}{\ms} 
        & \revision{}{\qty[mode=text]{3.7}{\ms}}
        & \qty[mode=text]{15.1}{\ms} 
        & \qty[mode=text]{5.2}{\ms} 
        & \qty[mode=text]{12.1}{\ms} \\
    & \pegaseSmall 
        & \qty[mode=text]{4.0}{\ms} 
        & \qty[mode=text]{5.3}{\ms} 
        & \revision{}{\qty[mode=text]{4.3}{\ms}}
        & \qty[mode=text]{18.4}{\ms} 
        & \qty[mode=text]{5.8}{\ms} 
        & \qty[mode=text]{51.5}{\ms} \\
    & \rteLarge 
        & \qty[mode=text]{5.9}{\ms} 
        & \qty[mode=text]{6.5}{\ms} 
        & \revision{}{\qty[mode=text]{6.3}{\ms}}
        & \qty[mode=text]{36.7}{\ms} 
        & \qty[mode=text]{9.5}{\ms} 
        & \qty[mode=text]{364.4}{\ms} \\
    & \pegaseLarge 
        & \qty[mode=text]{6.1}{\ms} 
        & \qty[mode=text]{7.0}{\ms} 
        & \revision{}{\qty[mode=text]{6.3}{\ms}}
        & \qty[mode=text]{93.2}{\ms} 
        & \qty[mode=text]{10.3}{\ms} 
        & \qty[mode=text]{913.5}{\ms} \\
    & \pegaseXLarge 
        & \qty[mode=text]{7.1}{\ms} 
        & \qty[mode=text]{8.2}{\ms} 
        & \revision{}{\qty[mode=text]{7.2}{\ms}}
        & \qty[mode=text]{561.2}{\ms} 
        & \qty[mode=text]{12.8}{\ms} 
        & \qty[mode=text]{1481.3}{\ms} \\
    & \gocXXL 
        & \qty[mode=text]{10.9}{\ms} 
        & \qty[mode=text]{11.7}{\ms} 
        & \revision{}{\qty[mode=text]{9.4}{\ms}}
        & \qty[mode=text]{444.2}{\ms} 
        & \qty[mode=text]{21.7}{\ms} 
        & \qty[mode=text]{4566.9}{\ms} \\
\bottomrule
\end{tabular}

        }\\
        \footnotesize{$^{\dagger}$with 200 gradient steps. $^{*}$solution time per instance (single thread).\\All ML inference times are for a batch of 256 instances.}
    \end{table}

    \begin{table}[!t]
        \centering
        \caption{Solving and inference time comparison (ED-R)}
        \label{tab:exp:ED-R:inference_time}
        \resizebox{\columnwidth}{!}{%
            \begin{tabular}{llrrrrr}
\toprule
Loss & System         &      DNN &    E2ELR  &   \revision{}{DeepOPF} &     DC3$^{\dagger}$ & GRB$^{*}$\\
\midrule
SL 
    & \ieeeSmall 
        & \qty[mode=text]{   3.9}{\ms} 
        & \qty[mode=text]{   6.5}{\ms}
        & \revision{}{\qty[mode=text]{4.6}{\ms}}
        & \qty[mode=text]{  16.5}{\ms} 
        & \qty[mode=text]{  12.6}{\ms} \\
    & \pegaseSmall 
        & \qty[mode=text]{   4.5}{\ms} 
        & \qty[mode=text]{   6.0}{\ms} 
        & \revision{}{\qty[mode=text]{4.8}{\ms}}
        & \qty[mode=text]{  18.9}{\ms} 
        & \qty[mode=text]{  56.5}{\ms} \\
    & \rteLarge 
        & \qty[mode=text]{   5.7}{\ms} 
        & \qty[mode=text]{  10.4}{\ms} 
        & \revision{}{\qty[mode=text]{6.2}{\ms}}
        & \qty[mode=text]{  36.1}{\ms} 
        & \qty[mode=text]{ 333.6}{\ms} \\
    & \pegaseLarge 
        & \qty[mode=text]{   6.3}{\ms} 
        & \qty[mode=text]{   7.7}{\ms} 
        & \revision{}{\qty[mode=text]{6.7}{\ms}}
        & \qty[mode=text]{  91.6}{\ms} 
        & \qty[mode=text]{1008.0}{\ms} \\
    & \pegaseXLarge 
        & \qty[mode=text]{   8.3}{\ms} 
        & \qty[mode=text]{  10.7}{\ms} 
        & \revision{}{\qty[mode=text]{8.8}{\ms}}
        & \qty[mode=text]{ 531.2}{\ms} 
        & \qty[mode=text]{1632.7}{\ms} \\
    & \gocXXL 
        & \qty[mode=text]{   9.3}{\ms} 
        & \qty[mode=text]{  11.1}{\ms}
        & \revision{}{\qty[mode=text]{10.6}{\ms}}
        & \qty[mode=text]{ 438.7}{\ms} 
        & \qty[mode=text]{4745.7}{\ms} \\
\midrule
SSL 
    & \ieeeSmall 
        & \qty[mode=text]{   3.9}{\ms}
        & \qty[mode=text]{   7.6}{\ms}
        & \revision{}{\qty[mode=text]{4.4}{\ms}}
        & \qty[mode=text]{  17.6}{\ms}
        & \qty[mode=text]{  12.6}{\ms} \\
    & \pegaseSmall 
        & \qty[mode=text]{   4.4}{\ms} 
        & \qty[mode=text]{   5.9}{\ms} 
        & \revision{}{\qty[mode=text]{4.7}{\ms}}
        & \qty[mode=text]{  19.1}{\ms} 
        & \qty[mode=text]{  56.5}{\ms} \\
    & \rteLarge 
        & \qty[mode=text]{   6.4}{\ms} 
        & \qty[mode=text]{  10.5}{\ms} 
        & \revision{}{\qty[mode=text]{6.7}{\ms}}
        & \qty[mode=text]{  37.3}{\ms} 
        & \qty[mode=text]{ 333.6}{\ms} \\
    & \pegaseLarge 
        & \qty[mode=text]{   7.1}{\ms} 
        & \qty[mode=text]{   8.3}{\ms} 
        & \revision{}{\qty[mode=text]{7.2}{\ms}}
        & \qty[mode=text]{  92.9}{\ms} 
        & \qty[mode=text]{1008.0}{\ms} \\
    & \pegaseXLarge 
        & \qty[mode=text]{   7.8}{\ms} 
        & \qty[mode=text]{   8.9}{\ms} 
        & \revision{}{\qty[mode=text]{7.9}{\ms}}
        & \qty[mode=text]{ 522.4}{\ms} 
        & \qty[mode=text]{1632.7}{\ms} \\
    & \gocXXL 
        & \qty[mode=text]{  10.2}{\ms} 
        & \qty[mode=text]{  12.4}{\ms} 
        & \revision{}{\qty[mode=text]{10.3}{\ms}}
        & \qty[mode=text]{ 435.8}{\ms} 
        & \qty[mode=text]{4745.7}{\ms} \\
\bottomrule
\end{tabular}

        }
        \footnotesize{$^{\dagger}$with 200 gradient steps. $^{*}$solution time per instance (single thread).\\All ML inference times are for a batch of 256 instances.}
    \end{table}


\subsection{Benefits of End-to-End Training}
\label{sec:results:benefits_e2e}

    \begin{table}[!t]
        \centering
        \caption{Comparison of optimality gaps (\%) with and without feasibility restoration (ED)}
        \label{tab:exp:ED:opt_gap:FFR}
        \resizebox{\columnwidth}{!}{\begin{tabular}{llrrrrrrrrrr}
\toprule
&& & \multicolumn{3}{c}{DNN} & \multicolumn{3}{c}{\revision{}{DeepOPF}} & \multicolumn{3}{c}{DC3}\\
\cmidrule(lr){4-6}
\cmidrule(lr){7-9}
\cmidrule(lr){10-12}
Loss & System         &    E2ELR & \multicolumn{1}{c}{--} & \multicolumn{1}{c}{FL} & \multicolumn{1}{c}{EP} & \multicolumn{1}{c}{--} & \multicolumn{1}{c}{FL} & \multicolumn{1}{c}{EP} & \multicolumn{1}{c}{--} & \multicolumn{1}{c}{FL} & \multicolumn{1}{c}{EP}\\
\midrule
SL   & \ieeeSmall     & \textbf{    1.42} &    69.55 &    36.33 &    36.37 &     \revision{}{2.81} &     \revision{}{2.81} &     \revision{}{2.81} &     3.03 &     3.03 &     3.03\\
     & \pegaseSmall   & \textbf{    0.74} &    48.77 &     3.98 &     3.94 &     \revision{}{7.78} &     \revision{}{5.24} &     \revision{}{5.28} &     2.80 &     2.44 &     2.44\\
     & \rteLarge      & \textbf{    1.35} &    55.13 &    21.28 &    21.41 &    \revision{}{28.23} &     \revision{}{1.89} &     \revision{}{1.89} &     3.68 &     3.36 &     3.36\\
     & \pegaseLarge   & \textbf{    0.38} &    76.06 &    34.61 &    34.65 &    \revision{}{33.20} &     \revision{}{1.94} &     \revision{}{1.99} &     1.25 &     1.24 &     1.24\\
     & \pegaseXLarge  & \textbf{    0.29} &    71.14 &    32.70 &    32.72 &    \revision{}{64.93} &    \revision{}{23.36} &    \revision{}{23.36} &     1.79 &     1.79 &     1.79\\
     & \gocXXL        & \textbf{    0.46} &   194.13 &    57.53 &    57.41 &    \revision{}{55.91} &    \revision{}{30.19} &    \revision{}{30.16} &     2.75 &     2.45 &     2.45\\
\midrule
SSL  & \ieeeSmall     & \textbf{    0.74} &    35.66 &     3.82 &     3.73 &     \revision{}{2.23} &     \revision{}{2.23} &     \revision{}{2.23} &     2.51 &     2.51 &     2.51\\
     & \pegaseSmall   & \textbf{    0.63} &    62.07 &     3.24 &     3.25 &    \revision{}{10.83} &     \revision{}{3.21} &     \revision{}{3.18} &     2.57 &     2.35 &     2.35\\
     & \rteLarge      & \textbf{    1.30} &    40.73 &    11.52 &    11.47 &    \revision{}{42.28} &     \revision{}{5.38} &     \revision{}{5.37} &     2.82 &     2.10 &     2.09\\
     & \pegaseLarge   & \textbf{    0.32} &    43.68 &     3.20 &     3.22 &    \revision{}{34.33} &     \revision{}{4.73} &     \revision{}{4.74} &     0.82 &     0.64 &     0.64\\
     & \pegaseXLarge  & \textbf{    0.21} &    57.58 &    20.59 &    20.59 &    \revision{}{60.12} &    \revision{}{18.85} &    \revision{}{18.84} &     0.84 &     0.81 &     0.81\\
     & \gocXXL        & \textbf{    0.39} &   108.91 &     7.89 &     7.89 &     \revision{}{8.39} &     \revision{}{3.06} &     \revision{}{3.06} &     0.72 &     0.62 &     0.62\\
\bottomrule
\end{tabular}
}
        % \footnotesize{All values are shifted geometric means with a shift of 1\%.}
    \end{table}

    \begin{table}[!t]
        \centering
        \caption{Comparison of optimality gaps (\%) with and without feasibility restoration (ED-R)}
        \label{tab:exp:ED-R:opt_gap:FFR}
        \resizebox{\columnwidth}{!}{\begin{tabular}{llrrrrrrrrrr}
\toprule
&& & \multicolumn{3}{c}{DNN} & \multicolumn{3}{c}{\revision{}{DeepOPF}} & \multicolumn{3}{c}{DC3}\\
\cmidrule(lr){4-6}
\cmidrule(lr){7-9}
\cmidrule(lr){10-12}
Loss & System         &    E2ELR & \multicolumn{1}{c}{--} & \multicolumn{1}{c}{FL} & \multicolumn{1}{c}{EP} & \multicolumn{1}{c}{--} & \multicolumn{1}{c}{FL} & \multicolumn{1}{c}{EP} & \multicolumn{1}{c}{--} & \multicolumn{1}{c}{FL} & \multicolumn{1}{c}{EP}\\
\midrule
SL   & \ieeeSmall     & \textbf{    1.52} &    75.06 &    30.47 &    30.49 &     \revision{}{2.80} &     \revision{}{2.80} &     \revision{}{2.80} &     2.94 &     2.94 &     2.94\\
     & \pegaseSmall   & \textbf{    0.74} &    47.84 &     2.52 &     2.50 &     \revision{}{7.50} &     \revision{}{4.79} &     \revision{}{4.79} &     2.97 &     2.34 &     2.34\\
     & \rteLarge      & \textbf{    1.82} &    70.57 &    30.20 &    29.90 &    \revision{}{46.66} &     \revision{}{2.63} &     \revision{}{2.51} &     3.49 &     3.32 &     3.29\\
     & \pegaseLarge   & \textbf{    0.38} &    81.19 &    41.34 &    41.40 &    \revision{}{30.84} &     \revision{}{1.90} &     \revision{}{1.92} &     1.29 &     1.29 &     1.29\\
     & \pegaseXLarge  & \textbf{    0.28} &    76.32 &    30.00 &    30.02 &    \revision{}{69.23} &   \revision{}{25.09} &    \revision{}{25.09} &     1.81 &     1.81 &     1.81\\
     & \gocXXL        & \textbf{    0.45} &   136.25 &    53.34 &    53.41 &    \revision{}{41.34} &   \revision{}{ 22.53} &    \revision{}{22.43} &     2.35 &     2.31 &     2.31\\
\midrule
SSL  & \ieeeSmall     & \textbf{    0.78} &    45.56 &     4.50 &     4.34 &     \revision{}{2.82} &     \revision{}{2.79} &     \revision{}{2.79} &     2.80 &     2.78 &     2.78\\
     & \pegaseSmall   & \textbf{    0.68} &    64.69 &     4.56 &     4.44 &     \revision{}{9.83} &     \revision{}{3.97} &     \revision{}{3.95} &     2.61 &     1.87 &     1.87\\
     & \rteLarge      & \textbf{    1.68} &    55.16 &     9.76 &     9.43 &    \revision{}{48.57} &     \revision{}{8.79} &     \revision{}{8.53} &     3.04 &     2.75 &     2.70\\
     & \pegaseLarge   & \textbf{    0.29} &    44.74 &     4.33 &     4.33 &    \revision{}{42.06} &     \revision{}{2.28} &     \revision{}{2.30} &     0.93 &     0.66 &     0.66\\
     & \pegaseXLarge  & \textbf{    0.19} &    61.28 &    21.35 &    21.32 &    \revision{}{65.38} &    \revision{}{19.64} &    \revision{}{19.64} &     0.91 &     0.89 &     0.89\\
     & \gocXXL        & \textbf{    0.33} &    93.91 &    10.00 &     9.98 &     \revision{}{9.47} &     \revision{}{2.58} &     \revision{}{2.58} &     0.71 &     0.64 &     0.64\\
\bottomrule
\end{tabular}
}
        % \footnotesize{All values are shifted geometric means with a shift of 1\%.}
    \end{table}

Tables \ref{tab:exp:ED:opt_gap:FFR} and \ref{tab:exp:ED-R:opt_gap:FFR}
further demonstrate the benefits of training end-to-end feasible
models: they report, for ED and ED-R problems, the optimality gaps
achieved by DNN, \revision{}{DeepOPF} and DC3 \emph{after applying a repair step at
  inference time}. Two repair mechanisms are compared: the proposed
Repair Layers (RL) and a Euclidean Projection (EP).  The tables also
report the mean gap achieved by E2ELR as a reference baseline.  The
results can be summarized as follows. First, the additional
feasibility restoration improves the quality of the initial
prediction.  This is especially true for DNN \revision{}{and DeepOPF}, which exhibited the
largest constraint violations (see Table \ref{tab:res:violation}):
optimality gaps are improved by a factor 2--20, but remain very high
nonetheless.  Second, the two repair mechanisms yield similar
optimality gaps.  For DC3, there is virtually no difference between RL
and EP.  Third, across all experiments, even after feasibility
restoration, E2ELR remains the best-performing model, with optimality
gaps 2--6x smaller than DC3. Table \ref{tab:exp:feasrec_time} compares
the computing times of the feasibility restoration using either the
repair layers (RL) or the euclidean projection (EP).  The latter is
solved as a quadratic program with Gurobi.  All benchmarks are
conducted in Julia on a single thread, using the
\texttt{BenchmarkTools} utility \cite{BenchmarkTools.jl-2016}, and
median times are reported.  The results of Table
\ref{tab:exp:feasrec_time} show that evaluating the proposed repair
layers is three orders of magnitude faster than solving the euclidean
projection problem.

    \begin{table}[!t]
        \centering
        \caption{Computing time of feasibility restoration using feasibility layers (FL) and Euclidean projection (EP).}
        \label{tab:exp:feasrec_time}
        \begin{tabular}{llrrr}
            \toprule
            Problem  & System           & \multicolumn{1}{c}{RL} & \multicolumn{1}{c}{EP} & Speedup \\
            \midrule
            ED    & \ieeeSmall       & \qty[mode=text]{  0.13}{ \us} & \qty[mode=text]{  0.45}{ \ms} & 3439x\\
                     & \pegaseSmall     & \qty[mode=text]{  0.55}{ \us} & \qty[mode=text]{  1.41}{ \ms} & 2572x\\
                     & \rteLarge        & \qty[mode=text]{  1.40}{ \us} & \qty[mode=text]{  3.75}{ \ms} & 2686x\\
                     & \pegaseLarge     & \qty[mode=text]{  2.37}{ \us} & \qty[mode=text]{  6.90}{ \ms} & 2911x\\
                     & \pegaseXLarge    & \qty[mode=text]{  6.42}{ \us} & \qty[mode=text]{ 20.71}{ \ms} & 3227x\\
                     & \gocXXL          & \qty[mode=text]{  5.67}{ \us} & \qty[mode=text]{ 17.87}{ \ms} & 3155x\\
            \midrule
            ED-R  & \ieeeSmall       & \qty[mode=text]{  1.06}{ \us} & \qty[mode=text]{  1.00}{ \ms} & 939x\\
                     & \pegaseSmall     & \qty[mode=text]{  4.58}{ \us} & \qty[mode=text]{  3.42}{ \ms} & 748x\\
                     & \rteLarge        & \qty[mode=text]{ 10.46}{ \us} & \qty[mode=text]{ 10.19}{ \ms} & 974x\\
                     & \pegaseLarge     & \qty[mode=text]{ 20.14}{ \us} & \qty[mode=text]{ 18.38}{ \ms} & 913x\\
                     & \pegaseXLarge    & \qty[mode=text]{ 42.67}{ \us} & \qty[mode=text]{ 60.73}{ \ms} & 1423x\\
                     & \gocXXL          & \qty[mode=text]{ 39.80}{ \us} & \qty[mode=text]{ 49.17}{ \ms} & 1236x\\
            \bottomrule
        \end{tabular}\\
        \footnotesize{Median computing times as measured by \texttt{BenchmarkTools}}
    \end{table}
    
